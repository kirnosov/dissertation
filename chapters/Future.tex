\chapter{FUTURE DIRECTIONS\label{future}}

\section{Muonic Systems}

A great advantage of this procedure developed in appendix \ref{apndx14} 
is that it is applicable to systems with more than three particles. 
For muon-catalyzed fusion, 
for example, the system of real interest is an electron attached 
to the three-body muonic systems treated in that work, 
or certain six-particle systems involving these four-body systems.

Such calculations should be expected in the nearest future.

\section{$\overline{p}dt$}

Another interesting example to consider is a system where
the negative particle has a mass of the same order as the
nuclei (e.g., $\overline{p}dt$). In Table \ref{pdtenergies}
we demonstrate the results obtained for $N$=0 and $N$=1 
states with no vibrational excitation. This system is known
to have only one weekly bound (ground) state 
(\cite{pdt_N1_1}, \cite{pdt_N1_2}) and can be modeled as a
$\overline{p}t$ complex polarized by a deuteron \cite{frolov_pdt}.

In our study we considered this system as a Coulomb
three-body system with unit charges and ignored the 
strong interactions within the system (which are clearly
very important for the extraction of the accurate physical data).
Nevertheless, such calculation is still a viable tool
to estimate to what extent adiabatic and non-adiabatic effects
are taken into account and also make conclusions on 
whether we can expect states to be bound.
 
For $N$=0 level we can compare our results to those obtained
by \cite{frolov_pdt} with the use of a model potential.
It can be seen that the value obtained using explicitly 
correlated Gaussians is lower by \NpdtD  protonic atomic
units. That should probably be attributed to a better 
description of the coupling between the particles 
in the present method than that provided by polarization
potential in \cite{frolov_pdt}. As for the $N$=1 state,
it can be compared to the dissociation energy of the
$\overline{p}dt$ ion, which is easily deduced directly
from the particles' masses. Our calculation suggests that this
system is unbound by \N1pdtD protonic a.u., which
is smaller than the current calculation inaccuracy.
Unfortunately, insufficient numerical precision 
does not allow to take the calculation further and show
the state boundness. 

In future calculations it would be interesting to describe 
strong interactions between the particles and perform the
computation with quadruple precision. 

\begin{table}[t]
\caption[$N$=0 and $N$=1 energies of the $\overline{p}dt$]
{$N$=0 and $N$=1 energies of the $\overline{p}dt$ expressed in
protonic atomic units. $N$=0 energy is compared to the value obtained by
\cite{frolov_pdt}, $N$=1 energy is shown to converge to a value
just above the dissociation energy of $\overline{p}dt$.  }
\centering
\begin{tabular}{r c   r c}
\hline\hline
$N$=0	~~&~~		~~&~~	$N$=1	~~&~~		~~\\
\# of ECGs	~~&~~	Energy	~~&~~	\# of ECGs	~~&~~	Energy	~~\\
\hline
500	~~&~~	-0.381\,190\,892\,903 &~~	250	~~&~~	-0.374\,803\,308\,724 \\
1000	~~&~~	-0.381\,190\,901\,332 &~~	500	~~&~~	-0.374\,803\,346\,392 \\
1500	~~&~~	-0.381\,190\,901\,991 &~~	750	~~&~~	-0.374\,803\,348\,201 \\
2000	~~&~~	-0.381\,190\,902\,159 &~~	1000	~~&~~	-0.374\,803\,348\,292 \\
\hline
\cite{frolov_pdt}~~&~~-0.381190899644 &~~Diss~~&~~-0.374803348316 \\
\hline
\end{tabular}
\label{pdtenergies}
\end{table}

\section{Trinuclear systems}

The diatomic basis functions \ref{N0function} can be easily extended 
to triatomic systems. Such an extension can be achieved by including 
in the Gaussian premultiplier powers of all internuclear distances. 
With that the basis functions are
\begin{equation}
\phi_k = r_1^{2m_k} r_2^{2n_k} r_{12}^{2p_k} \exp [r'(A_k \otimes I_3) r],
\label{3_1}
\end{equation}
where $r_1$ is the distance between the reference nucleus and
nucleus 2, $r_2$ is the distance between the reference nucleus and nucleus 3, 
and $r_{12}$ is the distance between nuclei 2 and 3. 
These functions should be good basis functions for expanding wave functions 
of rotationless (i.e., pure vibrational) states of triatomic systems with 
$\sigma$ electrons, for example, H$_3^+$ , H$_3$, or LiH$_2$. 
They should be able to effectively describe the three types of correlations, 
i.e., the electron-electron, nucleus-nucleus, and nucleus-electron correlations 
in the non-BO calculation.

However, the test calculations showed that the algorithms are not numerically 
stable because they involve oscillating series with finite number of elements 
whose values are large but have opposite signs. We have not been able to overcome 
this problem yet and thus this line of the development is presently on hold. 
In mean time, we have been searching for other alternative types of basis functions 
for use in triatomic non-BO calculations.

One type of functions which can potentially be effective in expanding non-BO 
triatomic wave functions are all-particle Gaussians multiplied by $sin$ and $cos$ 
functions dependent on squares of the internuclear distances:
\begin{equation}
\phi_k = f_1(a_1^k r_1^2 ) f_2(a_2^k r_2^2 ) f_{12}(a_{12}^k r_{12}^2 ) 
\exp [r'(A_k \otimes I_3) r],
\label{3_2}
\end{equation}
where $f_1(a_1^k r_1^2 )$ is either $sin(a_1^k r_1^2 )$ or $cos(a_1^k r_1^2 )$ 
and $f_2$ and $f_{12}$ having analogical forms.

That type of functions was tested in the calculations of bound pure vibrational 
states of a diatomic system with the interaction of the nuclei represented by a 
Morse potential. The test shows that, even though it takes more sin/cos-Gaussians 
than $r^m$-Gaussians to achieve similar precision of the energy, the former functions 
can effectively describe the vibrational states. Future work will involve implementation 
and testing of the sin/cos-Gaussians in all-particle non-BO calculations of the pure
vibrational states of diatomic systems. 
Subsequently, some small triatomics (e.g., H$_3^+$) will also be studied.