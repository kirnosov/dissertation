\chapter{NON-BORN-OPPENHEIMER CALCULATIONS\label{nonBO}}

In this chapter the author would like to discuss mostly the 
findings associated with the non-BO calculations of the binuclear
systems rovibrational spectra. These calculations differ from 
the non-BO atomic calculations, where the angular momentum 
excitation description is included with the corresponding 
spherical harmonic prefactor (see appendix \ref{apndx3} for an
example for L=3 state calculation, which author was actively 
involved with), 
due to the fact that the basis functions have to facilitate the 
description of the oscillatory nucleus-nucleus density.
Obviously, the shifted-center ECG functions used by author in
BO calculations (appendices \ref{apndx1} and \ref{apndx2}) are
not suitable since they are not eigenfunctions of the total
angular momentum squared operator.

Fully non-BO ECG calculations for diatomic molecules performed 
so far only concerned rotationless states, i.e., states represented 
by spherically symmetric wave functions (\cite{N0_1,N0_2}). 
In the calculations of those states the ECG function was multiplied 
by even powers of the internuclear distance, r$_1$ 
(the powers usually range from 0 to 250):
\begin{equation}
\phi_k^{N=0} = r_1^{p_k} \exp [-r'A_k r].
\label{N0function}
\end{equation}
This notation assumes that the internal coordinate frame is placed on a heavier
nucleus and the pseudoparticle one represents a second nucleus, while remaining
pseudoparticles have negative charge.

Therefore, function \ref{N0function} preserves spherical symmetry with respect
to the center of coordinate system while explicitly correlating all pseudoparticles
in the Gaussian exponent and introducing additional correlation factor for
the two nuclei. For the ground vibrational state, this additional factor combined 
with Gaussian exponent can very effectively generate a maximum around the vibrationally
averaged internuclear distance, thus effectively separating the two nuclei.
For the higher states, such a prefactor allows efficiently describe multiple node 
structure of the corresponding wave function.

Of course, integrals involving function \ref{N0function} are much more complicated 
than those derived for atomic case. One immediate consequence of that complexity 
is longer compute time and increasing requirements to programmatic implementation.
It should be also noted that now the algorithm has to perform not only continuous
parameters optimization, but also to chose good discrete parameters.

An important and interesting next step for the binuclear systems calculations
development was to find a way to efficiently include rotational excitation
in basis function and see what effect does rotational excitation have on
various expectation values. 

\section{Diatomic Molecules}

In order to describe the rotational excitation of the diatomic molecule,
a spherical harmonic prefactor can be added to the function \ref{N0function}:
\begin{eqnarray}
\phi_k^{N=1} &=& x_1 \cdot r_1^{p_k} \exp [-r'A_k r], \\
\label{N1function}
\phi_k^{N=2} &=& (x_1^2 + y_1^2 - 2z_1^2) \cdot r_1^{p_k} \exp [-r'A_k r].
\label{N2function}
\end{eqnarray}
Notice that in the basis functions above we make an assumption that 
pseudoelectron contributions to the total angular momentum would be 
negligible. The derivation of matrix elements and gradients for $N$=1
and $N$=2 is explained in great detail in appendices 
\ref{apndx4}, \ref{apndx8}, \ref{apndx9}, and \ref{apndx10}.

The effect of the angular excitation prefactor can be visualized
by the means of nucleus-nucleus correlation function presented 
in Figure \ref{1d}. In the picture on the left a symmetrically
symmetric correlation function corresponding to rotationless ($N$=0)
state is shown. The correlation function on the right corresponds to 
$N$=1 state and exhibits a dumbbell shape, which is an expected outcome
since we are using an atom-like Hamiltonian. Figure \ref{1d} also 
clearly displays an oscillatory behavior of the wave function. 

\begin{figure}[t]
\begin{center}
\includegraphics[width=2.5in]{../pics/4_full.png}
\includegraphics[width=2.5in]{../pics/4n0.png}
\end{center}
\caption{\label{1d}  2D correlation function plots for the 
$\nu$=3 of $N$=0 (on the left) and $N$=1 (on the right) rotational states.}
\end{figure}

\begin{table}[t]
\caption[The accuracy of the energy 
calculations for different
rovibrational levels of HD$^+$]{The accuracy of the energy 
calculations for different
rovibrational levels of HD$^+$. All values are in cm$^{-1}$.}
\centering
\def\arraystretch{1.}
\setlength{\tabcolsep}{.75em}
\begin{tabular}{c c c c | c c c c}
\hline \hline
v & $N$=0 & $N$=1 & $N$=2 & v & $N$=0 & $N$=1 & $N$=2  \\
\hline
0	&	0.00001	&	0.00236	&	0.00690	& 11	&	0.00012	&	0.00177	&	0.00500	\\
1	&	0.00003	&	0.00233	&	0.00680	& 12	&	0.00087	&	0.00242	&	0.00550	\\
2	&	0.00000	&	0.00223	&	0.00660	& 13	&	0.00024	&	0.00172	&	0.00470	\\
3	&	0.00003	&	0.00219	&	0.00640	& 14	&	0.00022	&	0.00171	&	0.00450	\\
4	&	0.00003	&	0.00213	&	0.00620	& 15	&	0.00087	&	0.00218	&	0.00480	\\
5	&	0.00005	&	0.00208	&	0.00610	& 16	&	0.00096	&	0.00205	&	0.00240	\\
6	&	0.00007	&	0.00205	&	0.00590	& 17	&	0.00168	&	0.00256	&	0.00410	\\
7	&	0.00007	&	0.00199	&	0.00570	& 18	&	0.00158	&	0.00222	&	0.00400	\\
8	&	0.00022	&	0.00206	&	0.00560	& 19	&	0.00127	&	0.00193	&	0.00350	\\
9	&	0.00010	&	0.00187	&	0.00540	& 20	&	0.00016	&	0.00071	&	0.00180	\\
10	&	0.00018	&	0.00189	&	0.00530	& 21	&	0.00047	&	0.00057	&	0.00080	\\
    &            &           &           & 22	&	0.00004	&	-0.00001	&	     \\
\hline
\label{table_energy}
\end{tabular}
\end{table}

From the table \ref{table_energy} it can be seen that the energy calculation
accuracy is decreasing as higher rotational states are considered, but still
stays on the level of 0.001 cm$^{-1}$ accuracy.
Relatively high calculation accuracy allowed us to perform a number
of high quality calculations of full spectra of diatomic molecules
(see appendices \ref{apndx3} and \ref{apndx8} for HD$^+$ calculations, 
\ref{apndx10} - for H$_2^+$, \ref{apndx11} - for H$_2$, \ref{apndx12} - 
for D$_2$, \ref{apndx7} and \ref{apndx9} - for HD), including 
a unique calculation of T$_2$ (appendix \ref{apndx13}). 

Since the nuclei are treated on equal footing with the electrons in the non-BO approach, 
the concept of the molecular geometry becomes elusive. The information about the molecular
structure can be extracted from the wave function as expectation values of the structural
parameters such as the interparticle distances. It was found (see appendices \ref{apndx5},
\ref{apndx7}, \ref{apndx8}, and \ref{apndx9}) that the internuclear distance increases
with rotational excitation. Moreover, charge asymmetry effects for heterogeneous molecules
also increase along with rotational number. 

While for HD molecule the charge asymmetry is small, for its molecular ion, HD$^+$,
uneven charge distribution effects are significant. The bond, which is covalent in low
vibrational levels, becomes purely ionic for top states. In the top $v$ = 22 state the 
electron is completely shifted from the proton to the deuteron and the average 
deuteron-electron distance of 1.552 a.u. is very close to its value of 1.500 a.u.
in the deuterium atom.

The charge asymmetry plays an important role in the interpretation of the results
obtained for the lifetimes of rotationless levels of HD$^+$. This work (appendix
\ref{apndx6}) high accuracy results for all vibrationally excited rotationless states 
were obtained for the first time. The oscillator strength values for the higher
states were diverging from the predicted pattern, which was a 
purely nonadiabatic effect arising from the electron localization on the
deuteron in the top three vibrational states. This effect cannot be described within 
the Born-Oppenheimer approximation, where the average position of the electron predicted 
by a calculation where the BO approximation is assumed is always in the middle of the bond. 
In non-BO calculations, such as the ones carried out in this work, this effect can be
adequately described. While insignificant for the most states, for the top three states, 
where the electron is much closer to the deuteron than to the proton, the BO and non-BO oscillation strengths predictions are expected to diverge.
The charge asymmetry in HD$^+$, which significantly increases in the top two 
$ν$ = 21 and $ν$ = 22 vibrational states, elongates their lifetime by up to 15$\%$.

\section{Exotic Systems}

After the successful application of the described method to electronic molecules,
a logical next step was to attempt to describe highly non-adiabatic systems, such
as muonic molecular ions. 

The attempt was  made to apply the procedure without modification, but, even though 
for $N$=0 states the agreement with the previous calculations was very good, 
for the $N$=1 there results were off from the previously reported values by 
as much as 10$^{−4}$ in muonic atomic units.
As the inaccuracy of the $N$=1 calculations increased with the increase of the 
mass of the lightest particle in the system, we attributed it to the deficiency 
of the ECG basis set used to represent the wave functions of the $N$=1 
rovibrational states. In the work described in great detail in appendix \ref{14} 
this deficiency was remedied.

The proper handling of the angular excitations of all particles involved in the 
system, as well as the coupling of these excitations, not only allowsed us to obtain 
much more accurate energies of the rovibrational states, but, more importantly,
opens a possibility of performing accurate calculations of a number of expectation 
values.

Even though we have been able to obtain quite accurate results for the $N$=1 states 
of some simple diatomic systems using basis functions \ref{N1function}, the question 
of the completeness of this set of functions has remained unanswered. The lack of 
completeness may results from only including in the basis set ECGs with the x$_1$ 
angular prefactor and not include ECGs with x$_i$, $i$ = 2 ,$...$, $n$ prefactors. 
For a molecule ECGs with such prefactors would describe the coupling between rotational 
excitations of the nuclei with excitations where the electrons and not the nuclei 
are excited to higher angular momentum states. The modified ECGs, which can describe 
this coupling have the form:
\begin{equation}
\phi_k^{(i)} = x_i \cdot r_1^{p_k} \exp [-r'A_k r],
\label{x_i_finc}
\end{equation}
where index $i$ ranges from 1 to the total number of pseudoparticles in the system.

With this modification, the accuracy on the order of 10$^{-12}$ a.u. or better was achieved
for ground states vibrational states with single rotational excitation of 
$dp\mu$, $tp\mu$ and $td\mu$ binuclear ions. For the vibrationally exited state 
of $td\mu$ the difference was up to several 10$^{-11}$ a.u.

The comparison of the dissociation energies of HD$^+$ calculated with the
inclusion of the basis functions \ref{x_i_finc} to those computed only with 
functions \ref{N1function} shows that explicit inclusion of angular excitations 
of the electron leads to a noticeable energy improvement for all states.
The improvement slowly decreases as the vibrational quantum number increases 
from 0 to 19 and then rapidly decreases for the three highest states. 
This behavior can be easily explained when the charge asymmetry in the 
vibrationally excited HD$^+$ molecular ion is considered.
As the electron moves closer to the deuteron it also moves closer to the 
center of mass of the system (i.e., the center of rotation for the molecule). 
This makes the moment of inertia of the rotating electron smaller and results 
in less energy being accumulated in its rotation.

Another interesting result of this study was that a contribution of the basis 
functions with angular prefactor $x_2$ could be directly evaluated.
For mounic systems the numbers are relatively consistent for all considered systems 
and range approximately between 25$\%$ and 28$\%$. It is interesting to note that for 
the system where the first excited ‘vibrational’ state with $N$=1 exists, i.e. $td\mu$, 
the number of ECGs with the $x_2$ prefactor was somewhat smaller than for the ground vibrational $N$=1 state. This seems to indicate that with the vibrational excitation 
the muon contribution to the total angular momentum decreases.
The percentage of $x_2$ functions for HD$^+$ was much smaller (around 7$\%$ on average)
and was consistent with the energy differences between two calculations.
For the highest state, were the basis set composed only of functions \ref{N1function}
did best, the percentage of functions with $x_2$ prefactor was smallest - only 4.8$\%$.  










