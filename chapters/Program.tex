\chapter{PROGRAMMATIC IMPLEMENTATION\label{prog}}

\section{Program Structure}

The algorithm of obtaining high accuracy wave functions
in the ECG basis set using analytic energy gradient
uses variational principle (\ref{var_principle})
and minimizes functional \ref{functional}.
The structure of the program is shown in Figure \ref{codeflow}.
The presented example has the specifications of non-BO calculation,
but it should be mentioned that the same program structure can be 
utilized for BO calculations with minor input modifications
(exclude masses, include molecular geometry). 

After the program initialization the system specifications
are read in. At this step particle masses need to be 
converted into the pseudoparticle reduced mass matrix,
charges should be stored properly, and a set of 
permutation operators must be constructed from the
symmetry operator.

Since the code is expected to be restarted multiple times
due to computer availability constraints and possible
failures, the current trial wave function expansion
is also stored in the input file. After the system
initialization the basis set can be read in, and 
Hamiltonian and overlap matrices should be computed.
In the next step, secular equation \ref{secular_equation}
needs to be solved to obtain linear expansion coefficients.

After the program processes all the provided data, 
so-called basis building and optimization program (BBOP)
starts. Despite the name, it can contain directives to
calculate single point expectation values and/or perform
various types of the basis set analyses.

\begin{figure}[H]
\begin{center}
\includegraphics[width=140mm]{../pics/CodeFlow.png}
\caption[Code Flowchart]{Code Flowchart. \label{codeflow}}
\end{center}
\end{figure}

The core of the algorithm is an optimization routine.
The user is free to chose different optimization regimes,
which have their advantages and disadvantages. The most
popular choice is to optimize one basis function at a time
and perform several optimization cycles on a given basis
set. In general, any subset of parameters may be chosen for 
optimization. The optimization is driven by 
Newton-Raphson optimization routine, which makes multiple
calls to matrix elements and gradients calculation 
functions.

After the existing basis set is sufficiently optimized, 
new basis functions can be added. One of the possible 
ways to generate it is to perturb the most contributing
function of the existing basis set. After the consecutive
optimization, this function can be added to the basis set.

\section{Performance Considerations}

An immediate observation one can make is that matrix elements
and gradients calculations performed multiple times and are
independent. For that reason a workload can be split between 
the multiple processors and calculations can be performed 
in parallel. 

In sight of the constantly changing computer architecture 
solutions a choice was made in favor of the code portability.
In a simple yet efficient approach communication between
processes is handled by means of Message Passing Interface (MPI).
The up-to-date basis set is maintained and
a copy of the calculated matrix is created on each processor.
After all the processes finish, the individual matrices are merged
on the master process.

It has to be noted that since shared memory solutions become 
available, the number of MPI processes can be reduced in favor 
of OpenMP threads. Such approach, while unlikely to result 
in calculation speed increase, would allow to limit memory 
usage and should be considered for the calculations where
basis set size approaches 15,000 functions.

Another aspect which would allow to increase the speed of individual
matrix element / gradient calculation would be to specify the system
parameters at the compile time. Providing compiler with the information
about exact matrix dimensions would allow to avoid dynamic memory
allocation, speed up functions invocation and improve in-function
caching. A way to do that is to start the code execution by 
running a Python script which reads an input file, updates 
settings file with the system of interest specifications,
recompiles the code and submits the job:

\lstinputlisting[language=Python]{../../compile_and_submit.py}





