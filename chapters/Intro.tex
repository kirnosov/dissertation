\chapter{INTRODUCTION\label{chap1}}

\section{Dissertation Structure}

This dissertation has an article-based format. All the manuscripts produced
during the candidacy are included as appendices. While each manuscript is 
self-contained and self-explanatory, in the context of this dissertation
they would be addressed as sources of the detailed description and more
specific analysis. The goal of this dissertation is to provide a reader 
with good understanding of the method applicability and its' advantages
and disadvantages in comparison to the other methods used for certain
classes of problems. 

In this chapter we will consider a general approach for accurate modeling
of the system wave function with a linear combination of explicitly 
correlated Gaussian functions (ECG).
In this approach the Born-Oppenheimer approximation is not assumed.
In order to relief the following 
chapters from the technical details, most of the mathematical concepts
are introduced here. The simplest form of the ECG function is considered
and its properties are discussed. 

In chapter \ref{nonBO} some applications of the approach to 
various systems with Coulomb interactions (atoms, molecules, $etc.$)
are described.
Modifications that need to be made to the basis function for each class
of systems are discussed.

Chapter \ref{prog} contains the description of the code used in the calculations
and discusses the nuances of the approach to each specific system implementation.
Moreover, some ways of increasing code performance and decreasing space 
requirements are considered.

Finally, chapter \ref{future} gives an overview of some avenues of research 
opened as a result of the work described in this dissertation.

\section{Coordinate Frame and Hamiltonian}

A system of $n+1$ particles with masses $M_i$ and charges $Q_i$ can be
described at any moment of time with $n+1$ coordinate vectors  

\begin{align}
    R_i &= 
    \begin{pmatrix}
           x_i \\           
           y_i \\
           z_i
    \end{pmatrix}
\end{align}

and $n+1$ momenta vectors

\begin{align}
    P_i &= 
    \begin{pmatrix}
           p_{x,i} \\           
           p_{y,i} \\
           p_{z,i}
    \end{pmatrix}.
\end{align}

The Hamiltonian operator for this system can be written as 

\begin{equation}
H = \sum_i^{n+1} \frac{P_i^2}{2M_i} + \sum_{i,j>i}^{n+1} \frac{Q_i Q_j}{R_{ij}},
\end{equation}

where

\begin{equation}
R_{ij} = \| R_i - R_j \|.
\end{equation}

It is easy to see that this laboratory frame Hamiltonian includes the center
of mass motion. In order to rigorously separate it out, a transformation 
of the coordinate system is performed. The transformation reduces the system
of $n+1$ particles with charges $Q_{i+1}$ and masses $M_{i+1}$, to an 
$n$ pseudo-particle system with charges $q_i = Q_{i+1}$ and reduced masses 
$\mu_i = \frac{M_1 M_{i+1}}{M_1 + M_{i+1}}$ moving in the
central field of the reference particle charge $q_0$. The mass of the
reference particle will be referred to as $m_0$.

Figure \ref{coord} illustrates the transition to the internal coordinate frame which 
has its center places on one of the particles. Normally for the reference particle
the most distinguishable (a nucleus in the atom) or the heaviest (deuteron
in HD$^+$) particle is chosen. The position of that particle with respect to the 
laboratory Cartesian coordinate frame is described by the vector $R_1$,
and other particles' coordinates are defined with respect to the reference particle:
$r_i = R_{i+1} - R_1$.

\begin{figure}[H]
\begin{center}
\includegraphics[width=150mm]{../pics/coordinates.png}
\caption[Coordinate frame transformation]{Coordinate frame transformation. \label{coord}}
\end{center}
\end{figure}

The internal Hamiltonian of the considered system now becomes:

\begin{equation}
H_{int} = \sum_{i=1}^n \frac{P^{2}_i}{2\mu_i} + \sum_{ i \neq j }^n \frac{P_i P_j}{2 m_0}
+ \sum_{i}^{n} \frac{q_i q_0}{\| r_i \|} + \sum_{i,j>i}^{n} \frac{q_i q_j}{\| r_i - r_j \|}.
\end{equation}

Substituting 

\begin{align}
P_i &= \frac{\nabla_{r_i}}{i} \\
r_{ij} &= r_j - r_i,
\end{align}

we obtain the final form of the internal Hamiltonian operator:

\begin{equation}
H_{int} = -\frac{1}{2} \left( \sum_{i=1}^n \frac{\nabla^2_{r_i}}{\mu_i} + \sum_{ i \neq j }^n \frac{\nabla_{r_i} \nabla_{r_j}}{m_0} \right)
+ \sum_{i}^{n} \frac{q_i q_0}{\| r_i \|} + \sum_{i,j>i}^{n} \frac{q_i q_j}{\| r_{ij} \|}.
\label{intH}
\end{equation}

It can be seen that the fist term of the internal Hamiltonian describes 
individual kinetic energies of the particles, while the third and the fourth
terms are used to take into account the Coulomb interactions between all
charges in the system. Second term is called a mass-polarization term and
originates from the choice of the internal coordinate system, which is
a non-inertial coordinate frame.

This Hamiltonian posses a number of important properties. It 
\begin{enumerate}
  \item describes all particles as quantum particles,
  \item accounts for all interactions,
  \item provides one-step variational approach,
  \item reduces problem dimensionality,
  \item is isotropic.
\end{enumerate}

It has to be mentioned that this Hamiltonian does not include the description 
of relativistic effects and thus should be more accurately called non-relativistic
internal Hamiltonian. The direct inclusion of additional terms to describe
relativistic ,as well as other interactions, is possible in principle.


\section{Explicitly Correlated Gaussian Functions}

Since the Schrodinger equation 
\begin{equation}
H \Psi = E \Psi
\end{equation}
is insoluble by current standards, we have to find a way
around this problem.

One possibility is to expand a wave function in a complete 
set of functions that obey the same boundary conditions
as the wave function:
\begin{equation}
\Psi (r) = \sum_k^{\infty} c_k \phi_k(r),
\end{equation}
where $c_k$ are the linear expansion coefficients.

Note that in the equation above the infinity symbol is used to 
communicate the completeness; the actual number of basis functions
does not have to be infinite. Nevertheless, the complete basis
set is not achievable for most cases and in practical calculations 
has to be truncated:

\begin{equation}
\Phi (r) = \sum_k^{K} c_k \phi_k(r).
\label{wf}
\end{equation}

The wave function representation (\ref{wf}) can be adequate for
an accurate description of a relatively simple and confined 
wave functions with a number of basis functions $K$ being
relatively small. The degree of completeness of such basis set 
may be evaluated by variational principle, which states that 
the expectation value of the Hamiltonian computed with the 
trial function is higher or equal to the true energy value:
\begin{equation}
E = \frac{\expval{H}{\Psi}}{\bra{\Psi}\ket{\Psi}} \leq 
\frac{\expval{H}{\Phi}}{\bra{\Phi}\ket{\Phi}}
\label{var_principle}
\end{equation}

In our work we have chosen to use the Gaussian functions 
of the following type as basis functions:
\begin{equation}
\phi_k = \exp \left[ - r' (A_k \otimes I_3) r \right],
\label{simple_gaussian}
\end{equation}
where $r$ is a vector of length $3n$ containing pseudo-particles'
coordinates in an internal coordinate frame ($r'=[x_1,y_1,z_1,...,x_n,y_n,z_n]$)
and $A_k$ is $n\times n$ symmetric matrix. Since the function (\ref{simple_gaussian})
must be square-integrable in order to be used for description of a bound state,
the matrix $A_k$ has to be positive definite. That can be achieved through the use
of Cholesky factorization $A_k=L_k L_k'$, where $L_k$ is a lower triangular matrix.
In the simplest case, all off-diagonal elements of $A_k$ matrix can be set to zero;
however, in order to describe the correlation effects between the particles,
they should also be non-zero. The resulting functions contain the correlation
explicitly and thus they are called Explicitly Correlated Gaussian functions.

It has to be noticed that according to the Pauli principle, the total wave function
which includes the spin degrees of freedom of a quantum system must either be symmetric 
or antisymmetric with respect to permutations of identical particles. 
Thus, our basis set expansion should incorporate this requirement. 
This puts a constraint on the symmetry of the basis 
functions that are used. 

In general, in order to build properly (anti)symmetric wave functions, we have to deal 
with both the spatial and spin coordinates. Due to the fact that the Hamiltonian of a
nonrelativistic Coulomb system does not depend on the spin of particles, it is possible 
to completely eliminate the spin variables from consideration. The projection operators
$Y$ for irreducible representations of the symmetric group (called Young operators) are 
obtained from their corresponding Young tableaux. Thus, our properly symmetrized 
trial wave function has the following form:
\begin{equation}
\Phi (r) = \sum_k^{K} c_k Y \phi_k(r).
\label{wf_perm}
\end{equation}



\section{General Method Description}

All the calculations have been performed within the framework of the 
Ritz variational method. 
The main idea of the variational method in the nonrelativistic 
quantum mechanics is based on 
the fact that the expectation value of the Hamiltonian of the system computed with a
trial wave function is always an upper bound to the exact ground state energy. 
This general property of the energy functional facilitates a way to obtain very accurate
approximations to the exact wave function by the optimization of the parameters, 
both linear and nonlinear, which the function comprises. This optimization is accomplished 
by the energy minimization. 

Therefore, in order to find the minimum of the Hamiltonian operator expectation value,
a secular equation 
\begin{equation}
Hc = ESc
\label{secular_equation}
\end{equation}
needs to be solved and energy functional 
\begin{equation}
E = min\frac{c'Hc}{c'Sc}
\label{functional}
\end{equation}
needs to be minimized with respect to the nonlinear parameters 
constituting basis functions.

In order to perform such minimization, the Hamiltonian and overlap matrices need
to be calculated. The individual matrix elements will be referred to as H$_{kl}$
and S$_{kl}$, respectively. The following $p$-dimensional Gaussian integral is used 
most often for this purpose:
\begin{equation}
\int_{-\infty}^{+\infty} \exp \left[ - r'Br + v'r \right] = 
\frac{\pi^{p/2}}{|B|^{1/2}} exp [-v'B^{-1}v].
\label{integral_general}
\end{equation}

Therefore, and overlap of simple ECG functions (\ref{simple_gaussian}) would be
\begin{equation}
\bra{\phi_k}\ket{\phi_l} = 
\frac{\pi^{p/2}}{|A_{k}+A_{l}|^{3/2}}.
\end{equation}

Later on, when we consider functions containing a prefactor, 
a generator function approach will be used. This approach involves
selecting a generator function which, when differentiated with respect to a 
parameter, produces the basis function with the prefactor. Since the parameter is 
independent on the coordinate variables, it is possible to integrate 
the generator functions using equation (\ref{integral_general}) and 
later take the derivative. This approach is described in details in
appendices \ref{apndx3}, \ref{apndx4}, and \ref{apndx14}.

A very important aspect of the calculations with ECG functions is that achieving 
high accuracy is possible only when the nonlinear exponential parameters of basis 
functions are extensively optimized. This process usually takes large amounts of 
computer time. To accelerate the basis set optimization in the ECG calculations, 
the analytic energy gradient determined with respect to the nonlinear parameters 
is employed. 
Evaluating the components of the gradient analytically as opposed to using finite 
differences of the energy allows to reduce the total number of operations
and avoid numerical precision issues to a certain extent.
This procedure for calculating the gradient is described for different cases 
in appendices \ref{apndx1}, \ref{apndx4}, and \ref{apndx14}.


